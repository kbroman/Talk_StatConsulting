\documentclass[12pt,t]{beamer}
\usepackage{graphicx}
\setbeameroption{hide notes}
\setbeamertemplate{note page}[plain]

% get rid of junk
\usetheme{default}
\beamertemplatenavigationsymbolsempty
\hypersetup{pdfpagemode=UseNone} % don't show bookmarks on initial view

% font
\usefonttheme{professionalfonts}
\usefonttheme{serif}
\usepackage{fontspec}
\setmainfont{Helvetica Neue}
\setbeamerfont{note page}{family*=pplx,size=\footnotesize} % Palatino for notes

% named colors
\definecolor{offwhite}{RGB}{249,242,215}
\definecolor{foreground}{RGB}{255,255,255}
\definecolor{background}{RGB}{24,24,24}
\definecolor{title}{RGB}{107,174,214}
\definecolor{gray}{RGB}{155,155,155}
\definecolor{subtitle}{RGB}{102,255,204}
\definecolor{hilight}{RGB}{102,255,204}
\definecolor{vhilight}{RGB}{255,111,207}
\definecolor{lolight}{RGB}{155,155,155}
%\definecolor{green}{RGB}{125,250,125}

% use those colors
\setbeamercolor{titlelike}{fg=title}
\setbeamercolor{subtitle}{fg=subtitle}
\setbeamercolor{institute}{fg=gray}
\setbeamercolor{normal text}{fg=foreground,bg=background}
\setbeamercolor{item}{fg=foreground} % color of bullets
\setbeamercolor{subitem}{fg=gray}
\setbeamercolor{itemize/enumerate subbody}{fg=gray}
\setbeamertemplate{itemize subitem}{{\textendash}}
\setbeamerfont{itemize/enumerate subbody}{size=\footnotesize}
\setbeamerfont{itemize/enumerate subitem}{size=\footnotesize}

% page number
\setbeamertemplate{footline}{%
    \raisebox{5pt}{\makebox[\paperwidth]{\hfill\makebox[20pt]{\color{gray}
          \scriptsize\insertframenumber}}}\hspace*{5pt}}

% add a bit of space at the top of the notes page
\addtobeamertemplate{note page}{\setlength{\parskip}{12pt}}

% a few macros
\newcommand{\bbi}{\vspace{24pt} \begin{itemize} \itemsep12pt}
\newcommand{\bi}{\begin{itemize}}
\newcommand{\ei}{\end{itemize}}
\newcommand{\ig}{\includegraphics}
\newcommand{\subt}[1]{{\footnotesize \color{subtitle} {#1}}}

% title info
\title{Statistical consulting}
\subtitle{My experiences}
\author{\href{http://www.biostat.wisc.edu/~kbroman}{Karl Broman}}
\institute{\href{http://www.biostat.wisc.edu}{Biostatistics \& Medical Informatics} \\[2pt] \href{http://www.wisc.edu}{University of Wisconsin{\textendash}Madison}}
\date{\href{http://www.biostat.wisc.edu/~kbroman}{\tt \scriptsize biostat.wisc.edu/{\textasciitilde}kbroman}}


\begin{document}

% title slide
{
\setbeamertemplate{footline}{} % no page number here
\frame{
  \titlepage
} }



\begin{frame}{My situtation}

\bbi
\item Applied statistician at a university
\bi
\item No formal consulting role
\item Do my best to answer colleagues' questions
\item Can always say no
\ei
\item Almost no experience with industry
\item Almost no experience with consulting companies
\ei

\end{frame}

\begin{frame}{Consulting vs. collaboration}

\bbi
\item {\color{hilight} Consulting}: short-term
\bi
\item Try to answer a specific question within a week or two
\ei
\item {\color{hilight} Collaboration}: long-term
\bi
\item Get to know the scientist's grander schemes
\ei

\ei

\end{frame}

\begin{frame}{My experience}

\bbi
\item Consulting class in graduate school at UC-Berkeley
\bi
\item Three times, with Andrew Gelman, Terry Speed, and David Freedman
\ei
\item Postdoc with a geneticist
\bi
\item Relatively isolated from other statisticians
\ei
\item Postdoc to faculty member
\bi
\item Steady stream of consulting experiences
\item From courses taught, referred by dept chair, other friends, software/papers
\ei
\ei

\end{frame}

\begin{frame}{Why do it?}

\bbi
\item Learn more biology
\item See new data
\item Think about new problems
\bi
\item And it's nice to get an easy one, sometimes
\ei
\item Be useful (help people)
\item Publications
\item Leads to long-term collaboration
\bi
\item Find out if you like the person
\ei
\item Could lead to new statistical methods research
\bi
\item But that's more often from longer-term collaborations
\ei
\ei
\end{frame}


\begin{frame}{The initial meeting}

\bbi
\item Listen
\item Verify that you're following
\item Don't be embarrassed to admit ignorance
\item Focus particularly on:
\bi
\item Scientific questions
\item Form of the data
\item Where the data came from
\ei
\item I'll usually delay giving specific advice to a second session
\ei
\end{frame}


\begin{frame}{Key difficulties}

\bbi
\item Admitting ignorance
\item Getting the scientist to back way up
\item ``I have a quick question\dots''
\item You don't have time
\item You see a big problem
\item Co-authorship on publications
\item They don't want anything fancy, and you think it's needed
\ei
\end{frame}

\begin{frame}{How to say no}

\bbi
\item Can you find someone to take over?
\item Give at least a rough guide
\item My two-week rule
\ei
\end{frame}

\begin{frame}{Disadvantages}

\bbi
\item You don't have time to do things properly
\item You don't know what you're getting into
\item You don't see long-term results
\item I hate power/sample size calculations
\ei
\end{frame}

\begin{frame}{Be self-sufficient}

\bbi
\item Learn multiple programming languages
\bi
\item R
\item C/C++
\item Perl/Ruby/Python
\ei
\item ``How would you like to receive the data?'' \\[12pt]
{\color{vhilight} ``In its current form.''}
\ei
\end{frame}

\begin{frame}{Time management}

\bbi
\item Don't let short-term things crowd out your long-term projects
\item Save big blocks of time for yourself.
\bi
\item I reserve a full day each week for my own work
\ei
\ei
\end{frame}

\begin{frame}{Summary}

\bbi
\item Statistical consulting is fabulous
\bi
\item Lots of new problems
\item Learn a lot of science
\item Help people
\ei
\item Don't need to know the answer
\bi
\item You just need to connect the scientist to the answer
\ei
\ei
\end{frame}


\end{document}
